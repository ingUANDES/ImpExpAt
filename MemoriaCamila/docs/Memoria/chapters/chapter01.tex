% cap1.tex

\chapter{Introducción}
\label{c1} % la etiqueta para referencias

Este trabajo pretende construir sobre trabajos previos de memoria como los de Ortiz y Tejero en torno a estudiar experimentos que midan atención en el marco de toma de decisiones con un énfasis en situaciones donde se haga valoración de medidas no pecuniarias como métrias medioambientales, sociales o de gobernanza corporativa (ASG o ESG por las siglas en inglés). La literatura sobre la importancia de este tipo de medidas de evaluación de inversión es vasta, ver por ejemplo \citeA{cornell_valuing_2020}. Sin embargo, no hay evidencia de medidas objetivas de atención puesta por los tomadores de decisión al momento de evaluar inversiones. Este trabajo pretende establecer un protocolo que permita testear evidencia objetiva sobre la atención puesta en este tipo de señales de información.

\section{Objetivos}

De esta forma, nos centraremos en un objetivo general que apunte a establecer una unificación de los trabajos teóricos de Ortiz y piloto experimental de Tejero.

\subsection{Objetivo general}

\begin{quote}
    Establecer un protocolo unificado para la recolección de datos pupilométricos en experimentos de toma de decisión con seguimiento ocular (eye-tracking).
\end{quote}

Este protocolo debe permitir comparar tres metodologías previamente implementadas bajo condiciones estandarizadas, asegurando la calidad, la integridad y la comparabilidad de los resultados. Asimismo, se busca generar hallazgos a partir del análisis comparativo de las metodologías y obtener un protocolo validado que sirva como guía para futuros estudios en áreas como educación, marketing, análisis del comportamiento y ciencias cognitivas.

\subsection{Objetivos específicos}


\begin{enumerate}
\item Definir objetivo especifico 1
\item obj esp 2
\end{enumerate}


\section{Alcances}

definir lo que se hace y lo que no se va a hacer en la memoria

\section{Metodología}


\begin{enumerate}
\item estudio de la tematica
\item presentación de los modelos econométricos

\end{enumerate}

\newpage
\section{Estructura del documento}

describir la estructura general del documento global



