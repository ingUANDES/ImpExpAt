\subsection{Experimental Test of Rational Inattention}
El paper "\textit{Experimental Tests of Rational Inattention}" realizado por Mark Dean y Nathaniel Neligh \cite{dean_experimental_2023}, aborda el tema de cómo los individuos enfocan su atención y esfuerzo cognitivo al tomar decisiones en entornos económicos, específicamente bajo el marco de la teoría de la inatención racional. La investigación se centra en el balance que hacen los individuos entre los costos de adquirir información y los potenciales beneficios derivados de hacer elecciones más informadas. A través de una serie de experimentos, los autores exploran cómo los sujetos ajustan su atención en función de cambios en los incentivos, así abordando el modelo de inatención racional.

Dean y Neligh consideran un tomador de decisiones (DM) que elige entre acciones, cuyos resultados dependen de un número finito de estados $\omega \in \Omega$ ocurre. La utilidad de la acción $a$ en el estado $\omega$ se denota por $u(a, \omega)$. Un problema de decisión se define por un conjunto de acciones disponibles $A$ y un estado previo del mundo $\mu \in \Delta(\omega)$, los cuales pueden ser elegidos por el experimentador. Los datos observados de un problema de decisión particular generan una función de elección estocástica dependiente del estado (SDSC), que describe la probabilidad de elegir cada acción disponible en cada estado del mundo. Para un problema de decisión $(\mu, A)$ usamos $P(\mu,A)$ para referirnos a la función SDSC asociada, siendo $P(\mu,A)(a|\omega)$ la probabilidad de que la acción $a \in A$ se haya elegido en el estado $\omega \in \Omega$. Se debe de tener en cuenta que una función SDSC también implica, a través de la regla de Bayes, un conjunto de "posteriores revelados", es decir, la distribución de probabilidad condicional sobre estados, $\gamma a$, asociada con cada acción $a \in A$ que se elige con probabilidad positiva.

Una de las contribuciones más significativas del estudio es su su cuestionamiento a la suposición comúnmente aceptada de que los costos de información son lineales en el modelo de Shannon. Los descubrimientos de la investigación dejan en evidencia que, si bien los sujetos sí ajustan su atención en respuesta a los incentivos de manera que es consistente con el modelo de inatención racional, el comportamiento observado no concuerda con un modelo donde los costos de información sean estrictamente lineales como lo planteado en el modelo de Shannon. Esto indica que la relación entre los costos de adquirir información y la cantidad de información adquirida puede ser más compleja y no lineal, requiriendo una generalización del modelo de Shannon para abordar adecuadamente este tema.

Finalmente, el estudio revela que una generalización del modelo de Shannon, que permite respuestas más flexibles a los incentivos y reconoce que algunos estados del mundo pueden ser más difíciles de distinguir que otros, proporciona un acercamiento más preciso a los datos experimentales. Esto es relevante porque sugiere que los modelos de toma de decisiones deben incorporar no solo la cantidad de información disponible sino también la calidad y la forma de presentación de esa información para predecir con precisión cómo las personas dirigirán su atención.

Las conclusiones del paper presentan la necesidad de repensar cómo modelamos los costos de información en la teoría económica. Asimismo, indica que el modelo de inatención racional, aunque en si ya es poderoso, puede beneficiarse significativamente de ser extendido para abordar las complejidades del procesamiento de información de los tomadores de decisión. Con esto, pueden desarrollar predicciones más precisas sobre el comportamiento y la toma de decisiones.


\subsection{Integration of ESG Information Into Individual Investors' Corporate Investment Decisions}
\citeA{park_integration_2022} llevaron a cabo una investigación que se centra en la integración de la información ASG en las decisiones de inversión de individuos en el mercado. El enfoque de este estudio es el uso del marco teórico UTAUT (\textit{Unified Theory of Acceptance and Use of Technology)} para examinar las razones y formas las cuales los inversionistas integran consideraciones ASG en sus estrategias de inversión. Este enfoque permite comprender los mecanismos subyacentes que afectan la integración de la información ASG, un área que ha sido relativamente poco explorada, especialmente desde el punto de vista de los inversionistas individuales.

Los autores señalan en su estudio que las dimensiones de expectativa de rendimiento, expectativa de esfuerzo, influencia social y condiciones facilitadoras son determinantes clave en la decisión de incorporar información ASG en la evaluación de inversiones. El estudio muestra cómo los inversores valoran la información ASG no solo por su posible impacto en el rendimiento financiero a largo plazo, sino también por su consonancia con valores personales, responsabilidad social y la manera que esta información se les presenta.

La importancia del rendimiento esperado como motivación para la integración de la información ASG es uno de los descubrimientos más significativos. Los inversionistas consideran que las inversiones con altos estándares ASG podrían ofrecer rendimientos superiores a largo plazo debido a su atención en la sostenibilidad y el manejo de riesgos asociados con aspectos sociales y ambientales, lo que a la larga traería estabilidad en la organización. La literatura usada en el paper respalda esta percepción al sugerir una correlación positiva entre las sólidas prácticas de ASG y el rendimiento financiero.

Sin embargo, el uso de medidas ASG presenta un desafío significativo: El procesamiento y análisis de la información ASG tiene un costo, esto es debido a la falta de estandarización en los informes ASG y a la variabilidad en la calidad y comparabilidad de los datos disponibles, lo que resulta en dificultades para interpretar la información por parte de los inversores. Por ende, es crucial estandarizar los informes ASG para reducir costos y facilitar la integración de criterios ASG en las decisiones de inversión.

\subsection{Inequality, Income, and Well-Being}
El trabajo de \citeA{decancq_chapter_2015}. proporciona un visión teorica de la conceptualización y medición del bienestar, abordando la complejidad inherente a la evaluación más allá de los indicadores económicos convencionales. En el documento, se aborda la pregunta de "¿qué es lo que deberíamos igualar?" en el contexto de la igualdad y el bienestar, lo que lleva a una discusión sobre el bienestar individual entendido a través de diversas lentes: Habilidades y desempeño, satisfacción y valor equivalente (indiferencia).

La función de bienestar es el centro del análisis, tomando en cuenta no solo los ingresos o consumos sino también aspectos más amplios de la vida que tienen un impacto directo en la calidad de vida de las personas. Esto muestra un cambio significativo hacia una evaluación del bienestar que considera múltiples dimensiones, como la salud, la educación y las relaciones sociales, reconociéndolos como elementos esenciales para el bienestar general de una persona. Es debido a esto que este estudio es de suma importancia para este proyecto en particular, donde se evaluará un aspecto no monetario (ASG) pero que trae otro tipo de beneficios.

El tratamiento de la "función de satisfacción" busca medir el bienestar subjetivo al tener en cuenta la evaluación que los individuos hacen de sus propias vidas. Reconocer la subjetividad inherente al bienestar es crucial, ya que distingue entre "utilidad experimentada" y "utilidad de decisión" para abordar las discrepancias entre cómo las personas anticipan sus decisiones y cómo realmente experimentan sus consecuencias. 

\subsection{Investing in Socially Responsible Mutual Funds}
El paper \textit{Investing in Socially Responsible Mutual Funds} de \citeA{geczy_investing_2003} trabaja el análisis de la inversión en fondos mutuos que siguen objetivos de inversión socialmente responsables (SRI). En el documento se evalúa el costo de incluir una restricción de SRI junto con la búsqueda del arrió de Sharpe más alto posible. 

El análisis revela que el costo percibido al agregar restricciones SRI depende fuertemente del modelo de valorización que el inversor utilice. Por ejemplo, para un inversor que confía plenamente en el modelo Modelo de Precios de Activos de Capital (CAPM) y no considera la habilidad del \textit{portaolio manager}, el costo de adherirse a inversiones SRI es mínimo, contabilizándose en tan solo unos pocos puntos base al mes. Sin embargo, el costo aumenta considerablemente, en al menos 30 puntos puntos base, para aquellos inversores que consideren modelos de valorización de activos que vinculen mayores retornos con exposiciones a factores de tamaño, valor y \textit{momentum}. Este costo de una restricción SRI en fondos mutuos es considerado una medida de perdida de rendimiento potencial que enfrentaran los inversores al limitar sus opciones únicamente fondos con objetivos SRI.