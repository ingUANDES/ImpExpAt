En el ámbito financiero la disponibilidad de información ha experimentado un aumento considerable a lo largo de los años, especialmente en el último siglo  debido a la digitalización. Para los inversionistas existen muchos factores que deben ser considerados a la hora de tomar una decisión, algunos de los más usados son el \textit{ROI}, \textit{ROE}, \textit{EBIT}, \textit{EBITDA}, entre muchos otros. Para estos individuos, informarse de toda la gama de estas métricas es costoso en tiempo y muy complicado de analizar y entender en profundidad, cómo consecuencia, han debido dejar de lado algunas y centrarse en las que más interés les traen según sus preferencias. 


\subsection{SRI (\textit{Socially Responsible Investments})}
Con el tiempo las métricas han evolucionado, apareciendo nuevas, incluso algunas no pecuniarias, es decir, no tienen un valor monetario directo sobre ellas. El concepto SRI por sus siglas en inglés o inversión socialmente responsable en español, hace referencia a inversiones que no solo toman en consideración las métricas de retornos monetarias comunes, sino se caracterizan por también considerar el impacto de la inversión en la sociedad. Es bajo esta idea que muchas métricas no pecuniarias han tomado un protagonismo importante, donde los algunos inversionistas, consideran la sostenibilidad en sus posiciones de inversión un factor relevante y a tomar en cuenta. 


Uno de los criterios más usados son los ESG (\textit{Enviromental, Social and Governance}) o, en sus siglas en español ASG (Ambiental, Social y Gobernanza) . Los factores ASG son los encargados de mostrar cómo una empresa se desempeña o impacta en su entorno. Los parámetros ASG son reportados por una amplia gama de empresas, en especial las de mayor tamaño, las cuales vienen marcando tendencia reportándolas periódicamente. Es más, en algunos lugares del mundo ya no es una opción dar a conocer las medidas ASG, sino una obligación debido a las nuevas regulaciones. Esto es claro en Europa, continente líder en este ámbito.

En adelante, se profundizara en varios aspectos claves de este trabajo, comenzando por los criterios ASG y cada uno de sus componentes para posteriormente abordar la inatención racional. Luego se introduce el marco teórico principal de la memoria, el cual se basa en literatura sobre inatención racional. Esta teoría da una base sólida para explorar como los tomadores de decisiones (o DM por el acrónimo de Decision Maker en inglés) procesan la información en un entorno financiero saturado de datos. Se abordará cómo esta teoría se relaciona en especifico con los indicadores ASG y su influencia en las posiciones tomadas por los inversionistas.

\subsection{Criterios ASG}
Los criterios ASG se dividen según sus siglas; en primer lugar el criterio ambiental, luego el criterio Social y por último el de Governanza. A continuación se explica en detalle qué abarca cada uno de estos criterios.

\subsubsection{Criterio Ambiental (A)}
El factor Ambiental dentro de los criterios ASG se enfoca en cómo una organización impacta y actúa frente al medio ambiente. Este criterio es de suma importancia dado a la creciente preocupación a nivel global de la sostenibilidad ambiental y el cambio climático. Algunos aspectos cruciales que se incluyen en este criterio son; emisiones de carbono, la gestión de recursos naturales, reducción de la contaminación generada en las actividades de la organización, gestión de residuos, entre otros. La relevancia de este factor ambiental radica en expresar cómo una empresa impacta en el medio ambiente y contribuye a la sostenibilidad a largo plazo. 

Los temas ambientales han tomado tal importancia que desarrollar soluciones para desafíos ambientales es crucial para proteger nuestro planeta y sentar las bases de un crecimiento económico sostenible a largo plazo. Debido a esto el sector financiero tiene un rol importante que desempeñar. Especialmente en movilizar el capital necesario para efectuar una transición hacia nuevas tecnologías y construir infraestructuras sostenibles \cite{jpmorgan_chase_2021_2021}.

\subsubsection{Criterio Social (S)}
El aspecto Social de los criterios ASG expresa las relaciones de una empresa con sus entorno social, es decir, con sus empleados, proveedores, clientes y comunidades. Dentro de este factor se abarcan temas como los derechos humanos , derechos del trabajo, diversidad, inclusión, participación en la comunidad, salud, practicas ética, entre varios más.

La importancia de este criterio radica en su capacidad de expresar como una organización se relaciona con sus grupos de interés. Un buen indicador social es crucial, ya que usualmente significa un buen ambiente laboral, lealtad con los empleados y clientes y en general una imagen pública positiva.

\subsubsection{Criterio Governanza (G)}
La forma en que una empresa se gobierna, sus prácticas empresariales y su impacto en accionistas y demás partes interesadas son abordados por el componente final de los criterios ASG denominado gobernanza. Se deben tener en cuenta componentes tales como diseño organizacional paritario e inclusivo, directrices éticas transparentes, estructura organizacional, remuneraciones justas además  de también resguardar adecuados derechos a todos quienes posean acciones. La relevancia de la gobernanza reside en su impacto en la durabilidad y los resultados futuros que obtenga una organización. Un buen gobierno corporativo basado en principios fuertes que garanticen transparencia es capaz no sólo generar seguridad para posibles inversores sino también prevenir situaciones no deseadas; asimismo ayuda a lograr un mejor enlace por parte del equipo directivo tanto con las metas empresariales como con aquellos que poseen acciones. Cuando las empresas incorporan prácticas fuertes de gobernanza se perciben como más confiables y responsables, aspectos que pueden impactar positivamente su posición competitiva en el mercado e incrementar la disposición para invertir en ellas.

Sin embargo, el estudio \textit{Sustainable Signals: Understanding Corporates Sustainability Priorities and Challenges} \cite{morgan_stanley_sustainable_2024} revela en su encuesta que las principales razones por la que las organizaciones adoptan practicas sostenibles no es por mejorar su funcionamiento en sí, sino que es crear valor para la empresa (50\%), cumplir con las normativas gubernamentales (48\%) y la obligación moral de hacer lo correcto para las personas y el planeta (47\%). Es recién en el cuarto lugar con un 46\% que mencionan que "la sostenibilidad es un reto importante para nuestro modelo de negocio, por lo que tenemos que reaccionar".


\subsection{Inatención Racional}

La inatención racional se estudia desde el punto de vista de la economía conductual y la psicología cognitiva. Se sustenta sobre la teoría de que frente a la abundancia de información disponible, los individuos son selectivos en cuanto a qué datos prestamos atención mediante una elección estratégica y deliberada, tomando en consideración solo algunos de los factores disponibles. Se requiere evaluar los costos involucrados en la obtención y procesamiento de información adicional frente a los posibles beneficios que esta pueda proporcionar. Asimismo, se basa en la noción de que un individuo tiene una capacidad cognitiva limitada, por lo tanto no puede procesar toda la información existente en su totalidad. Por consiguiente, las personas eligen poner atención a ciertos aspectos que consideran más relevantes o cruciales para lograr maximizar su beneficio; esto suele estar fundamentado en experiencias anteriores propias de cada individuo así como también de acuerdo a lo que creen y esperan. No se debe confundir esto con aleatoriedad en las decisiones sino que más bien es una combinación entre la necesidad de tomar decisiones correctamente ponderadas y la dificultad práctica para tener en cuenta toda el conjunto disponible. Esto aplica con la abundancia de información a la que se enfrentan los inversionistas, donde utilizar toda es muy difícil por lo que son selectivos en que métricas consideran a la hora de tomar sus decisiones.


\subsection{Motivación}
La motivación detrás de este experimento se centra en la intersección de dos temáticas de interés personal; las finanzas y la racionalidad en la toma de decisiones. Abordar estos dos tópicos, desde una perspectiva de finanzas responsables permite poder relacionar las dos áreas, en un ambiente en el que los inversionistas cada vez se encuentran más preocupados por el impacto de sus decisiones en la sostenibilidad y responsabilidad corporativa.
Asimismo, la información disponible sobre los indicadores medioambientales requieren que los tomadores de decisión entiendan e interpreten algo que a menudo es complejo e incompleto, desafío difícil de enfrentar, aún más cuando existe una sobrecarga de información, pero que sin duda les puede proporcionar una oportunidad tanto a corto como a largo plazo, por lo que abordar este tópico con un enfoque de inatención racional agrega valor a la investigación.
Por último, el estudio de temática es relevante al aumento de interés en las inversiones ASG, por ende, es esencial identificar qué rol toman los DM y si realmente consideran factores que no le traen un mayor beneficio económico.

\subsection{Problema}
El problema que aborda esta tesis radica no sólo en la necesidad de entender cómo los inversionistas utilizan la información ASG disponible, sino también entender como un factor no pecuniario influye en las decisiones que toma un inversionista, en especial en un ecosistema donde en la última década el compromiso con la inversión ética y sostenible ha aumentado. 
Asimismo los diferentes estándares en las métricas ASG y su complejidad hacen que la información disponible sobre estos criterios sean difíciles de entender y procesar, es en esta parte donde a los individuos se les presentará un escenario sencillo y fácil de entender para que puedan tomar una decisión sin que afecte las habilidades de interpretación de la métrica.
Este experimento sigue la linea de la tesis de Ezequiel Ortiz, sin embargo la implementación de eyetracker lleva esta investigación un paso mas allá, acercándose a datos reales y analizables. Mientras la tesis de Ortiz sentó las bases y una aproximación a los resultados, este experimento busca poder saber cuanta atención realmente le ponen los tomadores de decisión a los criterios ASG al momento de compáralo con el retorno del activo.