\documentclass[11pt,letterpaper]{article}
\usepackage{hyperref}
\usepackage[spanish]{babel}
\usepackage{multirow} 
\usepackage{multicol} 
\usepackage{subfig}
\usepackage{amsmath} % for \text{}
\usepackage{amssymb}
\usepackage{gensymb}
\usepackage{float}
\usepackage{mathrsfs} 
\usepackage{minted}
\usepackage{hyperref}

\usepackage[utf8]{inputenc}
% Control de color en tablas muy versátil.
\usepackage[table]{xcolor}
 % LaTeX


\title{Plantilla Informe}
  
% $Rev: 5 $
% PAQUETES %%%%%%%%%%%%%%%%%%%%%%%%%%%%%%%%%%%%%%%%%%%%%%%%%%%%%%%%%%%%%%%%%%%%
\usepackage{graphicx}
\usepackage{listings}
\usepackage{xcolor}
\usepackage{multicol}
\usepackage{anysize} %Permitir distintas medidas de margenes
\usepackage{framed}
\usepackage{fancyhdr}
\usepackage[spanish]{babel}
\usepackage[utf8]{inputenc}

\fancyhf{} 
\chead{ Proyecto de T\'itulo I \vspace{0.3cm}}
\lhead{\includegraphics[scale=0.1]{Log.png}}
\rfoot{\thepage}

\renewcommand{\footrulewidth}{0.25pt}

\addtolength{\headheight}{1.4cm}

% CONFIGURACION DE LISTINGS %%%%%%%%%%%%%%%%%%%%%%%%%%%%%%%%%%%%%%%%%%%%%%%%%%%
\newcommand{\AddUserKeywords}[1]{\lstset{morekeywords=[2]{#1}}}
\newcommand{\CodeSize}[1]{\lstset{basicstyle=#1\ttfamily}}
\newcommand{\CommentColor}{\color{green!60!black}}
\newcommand{\StringColor}{\color{red!70!black}}
\newcommand{\UserKeywordsColor}{\color{cyan!50!black}}
\newcommand{\KeywordsColor}{\color{blue}}

\lstdefinelanguage{CSharp}
{
	basicstyle=\small\ttfamily,
	keywordstyle=\KeywordsColor\textbf,
	keywordstyle=[2]\UserKeywordsColor,
	keywordstyle=[3]\StringColor,
	tabsize=2,
	morekeywords={abstract, as, base, bool, break, byte, case, 
		catch, char, checked,class, const, continue, decimal, 
		default, delegate, do, double, else, enum, event, explicit,
		extern, false, finally, fixed, float, for, foreach, get, goto, 
		if, implicit, in, int, interface, internal, is, lock, long,
		namespace, new, null, object, operator, out, override, 
		params, partial, private, protected, public, readonly, ref, return, 
		sbyte, sealed, set, short, sizeof, stackalloc, static, string,
		struct, switch, this, throw, try, typeof, true, uint, ulong, 
		unchecked, unsafe, ushort, using, value, virtual, volatile,
		void, while, where},
	morekeywords=[2]{Main,Console,String},
	morekeywords=[3]{@}, 
	commentstyle=\CommentColor,
	stringstyle=\StringColor,
	sensitive=true,
	morecomment=[l]{//},
	morecomment=[s]{/*}{*/},
	morestring=[b]",
	showstringspaces=false,
	aboveskip=0pt, 
	belowskip=0pt,
	mathescape=true
}

% Set as the default languaje
\lstset{language=CSharp}

\newcommand{\inguandesheader}{
	% Header Facultad Ingenieria Uandes			
	\includegraphics[scale=0.5]{uandes.pdf}\hspace*{\fill}
}

\newcommand{\evaluationtitle}[2]{
	% T\'itulo
	\begin{center}
	\vspace{1ex}\Large #1\\
	\vspace{1ex}\small #2
	\end{center}
}


\newenvironment{guideexercise}[3]{
	\noindent\textbf{#1}	
	\vspace{-0.3cm}
	\begin{framed}
		\noindent\textsl{Dificultad:} #2\\
		\textsl{Etiquetas:} #3\\	
				
}{
	\end{framed}
	\vspace{0.3cm}
}

\renewcommand{\baselinestretch}{1.5}

\pagestyle{fancy}
\begin{document}

%% COMIENZO PORTADA-----------------------------------------

\begin{titlepage}

\begin{center}
\includegraphics[scale=0.3]{Log.png}\\
% Incrementamos el interlineado:
\vspace{1.0cm} {\LARGE Implementación de un experimento de inatención racional con métricas de dilatación pupilar\\ Ingenier\'ia Civil Industrial UANDES\\

\vspace{1.5cm} \LARGE{Por:\\ Julio Tejero Caballero \\ Gu\'ia: \\ Sebasti\'an Cea}

\vspace{2.3cm}

\vspace{.5cm} \today

\end{center}
\end{titlepage}

 %%%% FIN PORTADA ------------------------------------------

\tableofcontents
\newpage


\section{Introducción}

\subsection{Criterios ASG}

 Actualmente la toma de decisiones financieras ha involucrado cada vez mas aspectos de distintas disciplinas, es más, hay algunos elementos no financieros que cada vez llaman más la atención de los tomadores de decisión (DM) a la hora de enfrentarse a una elección de activos, inversiones o proyectos. Estos aspectos trascienden el ámbito meramente económico, abarcando temas éticos, sociales, ambientales, entre otros; existiendo un criterio en particular que ha sido fuertemente considerado e incorporado por los DM en sus posiciones de inversión, los criterios ASG (o ESG en ingles). Estos, los cuales incorporan temas ambientales, sociales y gobernanza, son los factores no financieros más investigados a considerar al evaluar el atractivo de inversión de una empresa (Velte, 2019 CITA SECUNDARIA NO TENGO ACCESO AL PAPER DE VELTE).

 \subsubsection{Criterio ambiental (A)}
Los factores ambientales son aquellos que se relacionan con el impacto de una empresa, inversión o proyecto en el entorno natural. Esto engloba el manejo de los recursos naturales, emisiones de gases (como los gases de efecto invernadero), gestión de residuos, entre otros. Es decir, los indicadores ambientales se utilizan para evaluar como una organización actúa frente al la huella y el impacto que genera en el medio ambiente.

 \subsubsection{Criterio social (S)}
 Los factores sociales se enfocan en el comportamiento frente a las personas y la sociedad en general, aborda derechos laborales, condiciones de trabajo, relaciones con la comunidad, salud, entre otros, por lo que los indicadores sociales se usan para evaluar el comportamiento de una entidad frente a la sociedad y sus partes interesadas.

 \subsubsection{Criterio gobernanza (G)}
Este criterio hace referencia a la estructura, practicas y políticas de las organizaciones. Abarcando temas como ética empresarial, manejo de conflictos de interés, transparencia, entre otros. Por lo que los indicadores de gobernanza se utilizan para evaluar la calidad de gestión y la integridad de las entidades.


\subsection{Inatención racional}

La inatención racional ha sido estudiada desde diversas perspectivas y a la actualidad existen varios modelos que tienen como objetivo determinar el comportamiento de la atención del ser humano.


\subsection{Motivación}
La motivación detrás de este experimento se centra básicamente en la intersección de dos temáticas de interés personal; las fianzas y la toma de decisiones.\\
Abordar estos dos tópicos, desde una perspectiva de finanzas responsables permite poder relacionar las dos áreas, por un lado aspectos ambientales, sociales y de gobernanza y por el otro las decisiones de los DM, quienes cada vez se encuentran mas preocupados por el impacto de sus decisiones en la sostenibilidad y responsabilidad corporativa.
Asimismo, la información disponible sobre los indicadores medioambientales requieren que los tomadores de decisión entiendan e interpreten algo que a menudo es complejo e incompleto, desafío difícil de enfrentar pero que sin duda les puede proporcionar una oportunidad tanto a corto como a largo plazo.\\
La inatención racional es interesante agregarla al experimento ya que nos da una percepción de que datos influyen en la toma de decisión del inversores, en especial hoy en día en que existe mucha información disponible y los tomadores de decisión pueden ser selectivos en lo que prestan atención y lo que ignoran. 

\subsection{Problema}
El problema que aborda esta tesis radica no solo en la necesidad de entender cómo los inversores utilizan la información ASG disponible, sino también entender como un factor no pecuniario influye en las decisiones que toma un inversionista, \textbf{en especial} en un ecosistema donde el compromiso con la inversión ética y sostenible va en aumento. 

La aplicación de la teoría de inatención racional permite abordar el tema por un área relativamente inexplorada, por lo que esta implementación busca llenar ese vacío mediante la experimentación que permite observar de manera objetiva como los inversores interactúan con la información ASG en un terminal financiero, y en particular, cómo esa interacción se refleja en la dilatación pupilar y en el seguimiento ocular.\\

Este experimento sigue la linea de la tesis de Ezequiel Ortiz, sin embargo la implementación de eyetracker  lleva esta investigación un paso mas allá, acercándose a datos reales y analizables en un ecosistema de inversión real. Mientras la tesis de Ezequiel Ortiz sentó las bases y una aproximación a los resultados, este experimento busca llevar a la practica la teoría de la inatención racional en el contexto de la influencia de los indicadores ASG en las tomas de decisiones.


\section{Marco Teórico.}




\section{Diseño Experimental}



\section{Conslusiones}

\section{Revisión de Literatura}

\subsection{Experimental Test of Rational Inattention}

Los límites de información y error en la toma de decisiones ha llevado al desarrollo de varios modelos, tales como: Random Utility Model, Signal Detection Theory y modelos de “consideration set information\\

El comportamientode inatención racional se define por dos supuestos: la elección es óptima, condicionada a la información recibida; Asimismo el tomador de decisiones (DM) elige qué información recopilar para maximizar la utilidad de la elección posterior, neta de costos.\\

Consideramos a un tomador de decisiones (DM) que elige entre acciones, cuyos resultados dependen de cuál de un número finito de estados ω ∈ Ω ocurre. La utilidad de la acción a en el estado ω se denota por u(a, ω). Un problema de decisión se define por un conjunto de acciones disponibles A y un estado previo del mundo μ ∈ ∆(Ω), los cuales suponemos que pueden ser elegidos por el experimentador. Los datos observados de un problema de decisión particular generan una función de elección estocástica dependiente del estado (SDSC), que describe la probabilidad de elegir cada acción disponible en cada estado del mundo. Para un problema de decisión (μ, A) usamos P(μ,A) para referirnos a la función SDSC asociada, siendo P(μ,A)(a|ω) la probabilidad de que la acción a ∈ A se haya elegido en el estado ω ∈ Ω (donde no cause confusión, suprimiremos el subíndice en P). Tenga en cuenta que una función SDSC también implica, a través de la regla de Bayes, un conjunto de "posteriores revelados", es decir, la distribución de probabilidad condicional sobre estados, γa, asociada con cada acción a ∈ A que se elige con probabilidad positiva.\\

Los individuos claramente adoaptan su atención en respuesta de los incentivos\\

Hay semejanzas cualitativas de la data  y la condición LIP (Locally Invariant Posterior) que caracteriza uniformemente los modelos separables posteriormente.\\

Podemos pensar que hay tres características clave de los entornos informativos que podrían determinar el comportamiento: (1) la naturaleza del estímulo en el que se codifica la información, (2) la tecnología que tiene el tomador de decisiones para extraer información de ese estímulo (incluyendo cualquier restricción como límites de tiempo) y (3) la tarea para la cual se utilizará esa información.\\

\subsection{Integration of ESG Information Into Individual Investors' Corporate Investment Decisions}

Si bien el interes de los inversionistas es cada vez mayor, la investigación sobre la integración de la información ESG por parte de estos es insuficiente. Además, es necesario investigar los factores que influyen en los inversores individuales para integrar la información ESG.\\

La presencia de inversores individuales en el mercado de valores está aumentando. Los inversores individuales representan aproximadamente el 25\% de las actividades del mercado de valores debido a la volatilidad del mercado creada por el COVID 19, frente al 10\% de las actividades del mercado de valores en 2009 (Winck, 2022)\\

Los criterios ESG son los factores no financieros más importantes a considerar (junto con la información financiera) al evaluar el atractivo de inversión de una empresa (Velte, 2019; Lee et al., 2020).\\

En primer lugar, la inversión ESG promueve activamente prácticas de inversión éticas; en segundo lugar, las inversiones ESG se consideran un medio para mejorar el rendimiento de las carteras gestionadas y una forma de aumentar la rentabilidad y reducir el riesgo de la cartera (Broadstock et al., 2021).\\

Uno de los principales objetivos de la ISR a través de la integración de información ESG es la rentabilidad a largo plazo para los inversores (Eurosif, 2021). La expectativa de desempeño, el primer componente del modelo UTAUT para explicar la integración de la información ESG por parte de inversionistas individuales, se basa en la creencia de los usuarios potenciales de que se espera que la adopción de innovación o comportamiento innovador genere un mejor desempeño... Varios estudios en la literatura han sugerido que las empresas con buenas prácticas ESG tienen un mayor retorno de la inversión. Friede et al. (2015) realizaron un estudio sobre factores ESG/SRI y encontraron una relación positiva significativa entre el desempeño ESG y el desempeño financiero. Abate et al. (2021) muestra que la cartera de fondos compuesta por valores con alta calificación ESG obtuvo mejores resultados que sus contra-partes con baja calificación ESG.



\vspace{3cm}


Nota (escala 1 a 7): 
\vspace{3cm}
Observaciones



\centering Firma Guía



    






\newpage











































%ejemplo Plantilla



\end{document}
https://www.overleaf.com/project/62b5d06662bca87393f4c7d7