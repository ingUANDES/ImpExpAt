\documentclass[11pt,letterpaper]{article}
\usepackage{hyperref}
\usepackage[spanish]{babel}
\usepackage{multirow} 
\usepackage{multicol} 
\usepackage{subfig}
\usepackage{amsmath} % for \text{}
\usepackage{amssymb}
\usepackage{gensymb}
\usepackage{float}
\usepackage{mathrsfs} 
\usepackage{minted}
\usepackage{hyperref}

\usepackage[utf8]{inputenc}
% Control de color en tablas muy versátil.
\usepackage[table]{xcolor}
 % LaTeX


\title{Plantilla Informe}
  
% $Rev: 5 $
% PAQUETES %%%%%%%%%%%%%%%%%%%%%%%%%%%%%%%%%%%%%%%%%%%%%%%%%%%%%%%%%%%%%%%%%%%%
\usepackage{graphicx}
\usepackage{listings}
\usepackage{xcolor}
\usepackage{multicol}
\usepackage{anysize} %Permitir distintas medidas de margenes
\usepackage{framed}
\usepackage{fancyhdr}
\usepackage[spanish]{babel}
\usepackage[utf8]{inputenc}

\fancyhf{} 
\chead{ Proyecto de T\'itulo I \vspace{0.3cm}}
\lhead{\includegraphics[scale=0.1]{Log.png}}
\rfoot{\thepage}

\renewcommand{\footrulewidth}{0.25pt}

\addtolength{\headheight}{1.4cm}

% CONFIGURACION DE LISTINGS %%%%%%%%%%%%%%%%%%%%%%%%%%%%%%%%%%%%%%%%%%%%%%%%%%%
\newcommand{\AddUserKeywords}[1]{\lstset{morekeywords=[2]{#1}}}
\newcommand{\CodeSize}[1]{\lstset{basicstyle=#1\ttfamily}}
\newcommand{\CommentColor}{\color{green!60!black}}
\newcommand{\StringColor}{\color{red!70!black}}
\newcommand{\UserKeywordsColor}{\color{cyan!50!black}}
\newcommand{\KeywordsColor}{\color{blue}}

\lstdefinelanguage{CSharp}
{
	basicstyle=\small\ttfamily,
	keywordstyle=\KeywordsColor\textbf,
	keywordstyle=[2]\UserKeywordsColor,
	keywordstyle=[3]\StringColor,
	tabsize=2,
	morekeywords={abstract, as, base, bool, break, byte, case, 
		catch, char, checked,class, const, continue, decimal, 
		default, delegate, do, double, else, enum, event, explicit,
		extern, false, finally, fixed, float, for, foreach, get, goto, 
		if, implicit, in, int, interface, internal, is, lock, long,
		namespace, new, null, object, operator, out, override, 
		params, partial, private, protected, public, readonly, ref, return, 
		sbyte, sealed, set, short, sizeof, stackalloc, static, string,
		struct, switch, this, throw, try, typeof, true, uint, ulong, 
		unchecked, unsafe, ushort, using, value, virtual, volatile,
		void, while, where},
	morekeywords=[2]{Main,Console,String},
	morekeywords=[3]{@}, 
	commentstyle=\CommentColor,
	stringstyle=\StringColor,
	sensitive=true,
	morecomment=[l]{//},
	morecomment=[s]{/*}{*/},
	morestring=[b]",
	showstringspaces=false,
	aboveskip=0pt, 
	belowskip=0pt,
	mathescape=true
}

% Set as the default languaje
\lstset{language=CSharp}

\newcommand{\inguandesheader}{
	% Header Facultad Ingenieria Uandes			
	\includegraphics[scale=0.5]{uandes.pdf}\hspace*{\fill}
}

\newcommand{\evaluationtitle}[2]{
	% T\'itulo
	\begin{center}
	\vspace{1ex}\Large #1\\
	\vspace{1ex}\small #2
	\end{center}
}


\newenvironment{guideexercise}[3]{
	\noindent\textbf{#1}	
	\vspace{-0.3cm}
	\begin{framed}
		\noindent\textsl{Dificultad:} #2\\
		\textsl{Etiquetas:} #3\\	
				
}{
	\end{framed}
	\vspace{0.3cm}
}

\renewcommand{\baselinestretch}{1.5}

\pagestyle{fancy}
\begin{document}

%% COMIENZO PORTADA-----------------------------------------

\begin{titlepage}

\begin{center}
\includegraphics[scale=0.3]{Log.png}\\
% Incrementamos el interlineado:
\vspace{1.0cm} {\LARGE Implementación de un experimento de inatención racional con métricas de dilatación pupilar\\ Ingenier\'ia Civil Industrial UANDES\\

\vspace{1.5cm} \LARGE{Por:\\ Julio Tejero Caballero \\ Gu\'ia: \\ Sebasti\'an Cea}

\vspace{2.3cm}

\vspace{.5cm} \today

\end{center}
\end{titlepage}

 %%%% FIN PORTADA ------------------------------------------

\tableofcontents
\newpage


\section{Introducción}

\subsection{Criterios ASG}

 Actualmente la toma de decisiones financieras ha involucrado cada vez mas aspectos de distintas disciplinas, es más, hay algunos elementos no financieros que cada vez llaman más la atención de los tomadores de decisión (DM) a la hora de enfrentarse a una elección de activos, inversiones o proyectos. Estos aspectos trascienden el ámbito meramente económico, abarcando temas éticos, sociales, ambientales, entre otros; existiendo un criterio en particular que ha sido fuertemente considerado e incorporado por los DM en sus posiciones de inversión, los criterios ASG (o ESG en ingles). Estos, los cuales incorporan temas ambientales, sociales y gubernamentales, son los factores no financieros más investigados a considerar al evaluar el atractivo de inversión de una empresa (Velte, 2019 CITA SECUNDARIA NO TENGO ACCESO AL PAPER DE VELTE).

 \subsubsection{Criterio ambiental (A)}

 \subsubsection{Criterio social (S)}

 \subsubsection{Criterio gubernamental (G)}

\subsection{Inatención racional}

La inatencion racional ha sido estudiada desde diversas perspectivas y a la actualidad existen varios modelos que tienen como objetivo determianar el comportamiento de la atención del ser humano.


\subsection{Motivación}
La motivación detras de este experimento se centra basicamente en la intersección de dos tematicas de interés personal; las fianzas y la toma de decisiones.\\
Abordar estos dos tópicos, desde una perspectiva de finanzas responsables permite poder relacionar las dos areas, por un lado aspectos ambientales, sociales y de gobernanza y por el otro las decisiones de los DM, quienes cada vez se encuentran mas preocupados por el impacto de sus decisiones en la sostenibilidad y responsabilidad corporativa.
Asimismo, la información disponible sobre los indicadores medioambientales requieren que los tomadores de decision entiendan e interpreten algo que a menudo es complejo e incompleto, desafío dificil de enfrentar pero que sin duda les puede proporcionar una oportunidad tanto a corto como a largo plazo.\\
La inatención racional es interesante agregarla al experimento ya que nos da una persepción de que datos influyen en la toma de decsion del inversores, en especial hoy en día en que existe mucha información disponible y los tomadores de decisión pueden ser selectivos en lo que prestan atención y lo que ignoran. 

\subsection{Probelma}
El problema que aborda esta tesis radica no solo en la necesidad de entender cómo los inversores utilizan la información ASG disponible, sino tambien entender como un factor no precuniario influye en las decisiones que toma un inversionista, \textbf{en especial} en un ecosistema donde el compromiso con la inversión ética y sostenible va en aumento. 

La aplicación de la teoría de inatención racional permite aboradar el tema por un area relativamenta inexplorada, por lo que esta implememtación busca llenar ese vacío mediante la expermientación que permite observar de manera objetiva como los inversores interactúan con la información ASG en un terminal financiero, y en particular, cómo esa interacción se refleja en la dilatación pupilar y en el seguimiento ocular.\\

Este experimento sigue la linea de la tesis de Ezequiel Ortiz, sin embargo la implementación de eyetracker  lleva esta investigación un paso mas allá, acercandose a datos reales y analizables en un ecosistema de inversión real. Minetras la tesis de Ezequiel Ortiz sentó las bases y una aproximación a los resultados, este expermiento busca llevar a la practica la teoria de la inatención racional en el contexto de la influecia de los indicadores ESG en las tomas de decisiones.


\section{Marco Teórico.}


\section{Diseño Experimental}



\section{Conslusiones}




\vspace{3cm}


Nota (escala 1 a 7): 
\vspace{3cm}
Observaciones



\centering Firma Guía



    






\newpage











































%ejemplo Plantilla



\end{document}
https://www.overleaf.com/project/62b5d06662bca87393f4c7d7